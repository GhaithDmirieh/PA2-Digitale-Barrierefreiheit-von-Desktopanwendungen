\section{Einleitung}

\subsection{Aufbau der Arbeit und methodisches Vorgehen}
Um die Kriterien für die Barrierefreiheit in Desktopanwendungen zu erarbeiten, wird als erstes die digitale Barrierefreiheit definiert und es wird auf die Nachteile fehlender Barrierefreiheit in Software eingegangen. Außerdem wird die Zielerreichung dieser Arbeit in \cref{subsec: Schaffung von Barrierefreiheit in Desktopanwendungen und Abgrenzung der Arbeit} dargestellt. Danach wird das Unternehmen und die Softwareabteilung des Unternehmens vorgestellt. Darüber hinaus werden wichtige Begriffe erklärt. Es wird auch begründet, aus welchem Grund auf die digitale Barrierefreiheit geachtet werden muss. Welche Rolle die digitale Barrierefreiheit in der Software-Entwicklung spielt und wieso zwischen Barrierefreiheit und Behindertengerecht unterschieden werden muss, wird durch Betrachtung der Ist-Zustand klar gemacht. Vom \cref{subsec:Prinzipien fuer Barrierefreiheit} bis \cref{subsec: Normen der WCAG 2.1} werden Normen der digitalen Barrierefreiheit von der "`World Wide Web Consortium"' \footnote{The World Wide Web Consortium (W3C) \cite{w3c}} sowie die Unterschiede zu den \ac{BITV} präsentiert. Schließlich werden Kriterien der Standards auf die Umsetzbarkeit in Dialogen der Desktopanwendungen hin untersucht.

\subsection{Ein Blick in die Vergangenheit der Barrierefreiheit}
Der Begriff Barrierefreiheit wurde seit den 1970er-Jahren von Menschen mit Behinderungen eingefordert und wird in Deutschland als Synonym für den englischen Bergriff \textit{Accessibility} betrachtet. Es beutet, dass Menschen mit Behinderungen den Zugang zu unterschiedlichen Infrastrukturen haben. Der Begriff Barrierefreiheit ist im Bereich der Nutzung von Gebäuden oder Verkehrsmitteln im Sinne von \textit{Zugang} wörtlich zu nehmen, da die Rede vom physischen Zugang ist. Sobald man von der Formation und der Kommunikation redet, erweitert sich der Begriff vom physikalischen Zugang zu den Inhalten, den Bedingungen und dem Verständnis.\footnote{Accessibility über Desktopanwendungen hinaus–Barrierefreiheit \cite{buhler2017accessibility}} Der Begriff Barrierefreiheit bekam im Jahr 2002 eine moderne gesetzliche Definition, die im Jahr 2016 angepasst wurde: "`Barrierefrei sind bauliche und sonstige Anlagen, Verkehrsmittel, technische Gebrauchsgegenstände, Systeme der Informationsverarbeitung, akustische und visuelle Informationsquellen und Kommunikationseinrichtungen sowie andere gestaltete Lebensbereiche, wenn sie für Menschen mit Behinderungen in der allgemein üblichen Weise, ohne besondere Erschwernis und grundsätzlich ohne fremde Hilfe auffindbar, zugänglich und nutzbar sind."'\footnote{Vgl. Accessibility über Desktopanwendungen hinaus–Barrierefreiheit S. 501 \cite{buhler2017accessibility}}

Diese Definition betrachtete die Nutzbarkeit als Hauptanliegen, jedoch die Wandlung zur Informations- und Wissensgesellschaft war nicht mehr zu ignorieren und damit kam der Anwendungsbereich der \ac{IKT} in das Blickfeld. Das Nichtnutzen dieser alternativen Zugangsmöglichkeiten hat Menschen mit Behinderungen, die nicht online gehen können getroffen. Diese Zielgruppe war benachteiligt oder sogar ausgeschlossen. Fügt man nun die \ac{IKT} zu der vorherigen Definition hinzu, wird der Zusammenhang mit der \textit{Zugänglichkeit} unmittelbar deutlich.\footnote{Accessibility über Desktopanwendungen hinaus–Barrierefreiheit \cite{buhler2017accessibility}}

\subsection{Die Bedeutung der digitalen Barrierefreiheit}
\label{subsec: Die Bedeutung der digitalen Barrierefreiheit}

Digitale Barrierefreiheit bedeutet, dass Menschen das Internet nutzen können, dessen Inhalte wahrnehmen, verstehen, navigieren und mit ihm integrieren können. In dieser Hinsicht kommen Menschen mit Behinderungen stärker in den Fokus.\footnote{Gesellschaft für digitale Bildung \cite{GFDB}} "`Laut der Aktion Mensch-Studie nutzen Menschen mit Behinderung das Internet öfter als Menschen ohne Behinderung. Elektronische Interaktionen sind demnach von besonderer Bedeutung, weil sie Zugang zu bestimmten Angeboten überhaupt erst ermöglichen."'\footnote{Vgl. Gesellschaft für digitale Bildung \cite{GFDB}}. Aber unter der digitalen Barrierefreiheit betrachtet man mehr als die körperlichen Einschränkungen wie z.B. Schwerhörigkeit oder Blindheit, wie beispielsweise die fehlende Möglichkeit, einen Ton abzuspielen. Ebenso verhält es sich mit der Kontrasterkennung bestimmter Farben oder der Gestaltung der Inhalte. Außerdem sind viele digitale Angebote, die oft \textbf{exklusiv} online verfügbar sind, nur schwer zu bedienen. Anders gesagt, sie sind nicht \textit{nutzerfreundlich}. Diese Barrieren sind alles Faktoren, die für alle Menschen eine Rolle spielen und auch ganze Zielgruppen ausschließen können.\footnote{Gesellschaft für digitale Bildung \cite{GFDB}} Es ist also zu unterscheiden zwischen Barrierefreiheit und Behindertengerecht, was in \cref{subsec:Barrierefreiheit und Behindertengerecht} ausführlich erklärt ist. Um nun Barrierefreiheit in den digitalen Software zu schaffen, sind vier Prinzipien in jeder \ac{GUI} zu erfüllen. Diese sind laut \ac{WCAG} 2.0 \footnote{Web Content Accessibility Guidelines 2.0 \cite{caldwell2008web}}:

\begin{itemize}
	\item Wahrnehmbar
	\item Bedienbar
	\item Verständlich
	\item Robustheit
\end{itemize}

Auf die einzelnen Begriffe wird in \cref{subsec:Barrierefreiheit und Behindertengerecht} eingegangen.

\subsection{Fehlende Barrierefreiheit in Desktopanwendungen}
\label{subsec: Fehlende Barrierefreiheit in Desktopanwendungen}

Viele Zielgruppen werden die Software nicht nutzen können und alle anderen Nutzer haben es nicht leicht zurechtzukommen. Das sind die ersten Folgen der fehlenden Barrierefreiheit in jeder Software. "`Um die Herausforderungen an Barrierefreiheit zu verstehen, muss man sich die Diversität der Nutzerinnen und Nutzer vor dem Hintergrund der Variation der Geräte und Anwendungen vor Augen führen. Diese ist schon ohne die Berücksichtigung von Menschen mit Behinderungen sehr groß und unterscheidet sich im Hinblick auf z. B. die technische Sozialisation (,,digital native“, ,,digital illiterate“), den Bildungsgrad (Analphabetismus, akademische Bildung), die Nutzung (beruflich, die Nutzung für Information und Kommunikation, für Spiele, für Gesundheitsanwendungen) sowie die Umgebung und die Situation der Nutzung usw. Schon hier wird deutlich, dass unterschiedlichste Anforderungen berücksichtigt werden müssen, um alle Bedarfe abzudecken."'\footnote{Vgl. Accessibility über Desktopanwendungen hinaus–Barrierefreiheit S. 504 \cite{buhler2017accessibility}}

Zusammengefasst sind folgende Nachteile der digitalen Barrieren in jeder Software zu beachten:
\vspace{1em}

\begin{itemize}
	\item Es wird keine größere Reichweite erreicht
	\item Weniger Performance, denn nicht alle Menschen können die Software nutzen
	\item Weniger Erfolg, denn Menschen, die die Software regelmäßig verwenden, werden diese auch nicht weiter empfehlen, wenn sie selber Schwierigkeiten damit haben
\end{itemize}

\subsection{Schaffung von Barrierefreiheit in Desktopanwendungen und Abgrenzung der Arbeit}
\label{subsec: Schaffung von Barrierefreiheit in Desktopanwendungen und Abgrenzung der Arbeit}

Ziel dieser Arbeit ist das Untersuchen der Machbarkeit der Schaffung von digitaler Barrierefreiheit in der aktuellen Software sowie Regeln dafür festzulegen, wie in jeder Software der AKG Software Consulting GmbH digitale Barrierefreiheit umgesetzt werden kann. Bei der Entwickelung neuer Software soll auf diese Regeln geachtet werden, um digitale Barrieren so gut wie möglich zu vermeiden. Es wird untersucht, wie man die digitale Barrierefreiheit erreicht, wofür die Normen der \ac{WCAG} 2.0 bzw. der \ac{WCAG} 2.1 in Frage kommen, die auch möglicherweise nicht komplett umsetzbar sein können. Außerdem wird auf die Vor- und Nachteile, die einem Entwickler begegnen werden, eingegangen. Zu der aktuellen Software bietet das Unternehmen eine Dokumentation an, die jeden Dialog der Software beschreibt. Dazu gehört eine Beschreibung von der Struktur, inhaltlichen Informationen, Fachbegriffen usw.. Diese Dokumentation ist webbasiert und wird bei der Untersuchung der Barrierefreiheit in dieser Projektarbeit nicht betrachtet.