\section{Anforderungen an die Software}

\subsection{Kriterienvergleich der Barrierefreiheit zwischen Web- und Desktopanwendungen}

Die Normen der \ac{WCAG} 2.0 bzw. der \ac{WCAG} 2.1 sind an der ersten Stelle für entweder Inhalte von Webseiten oder Inhalte der webbasierten Anwendungen orientiert. Allerdings kommen diese Normen auch im Bereich Mobile-Anwendungen zum Einsatz.\footnote{The World Wide Web Consortium (W3C) \cite{w3c}}

Auf der anderen Seite haben die Normen der \ac{BITV} die Regel außer Webseiten und Mobile-Anwendungen auf jede grafische Programmoberfläche verallgemeinert, die einer der folgenden Bedingungen betrifft:\footnote{Die Barrierefreie-Informationstechnik-Verordnung 2.0 \cite{BITV}}

\begin{itemize}
	\item Die Software ist für Angebote, Anwendungen oder Dienste, die elektronisch unterstützte Verwaltungsabläufe, einschließlich der Verfahren zur elektronischen Vorgangsbearbeitung 
	und elektronischen Aktenführung, gedacht.
	\item Die Software wird zur Nutzung an öffentlichen Stellen bereitgestellt.
\end{itemize}

Schlossfolgerung steht offiziell für Desktopanwendungen, die nicht an öffentlichen Stellen genutzt wird und die nicht für die Nutzung von allen Menschen gedacht ist, noch keine feste Regeln für digitale Barrierefreiheit. Das betrifft Unternehmenssoftware, welche nur intern im Unternehmen genutzt wird. Aus diesem Grund wird einen Vergleich zwischen Web- und Desktopanwendungen gezogen, um herauszufinden, wie viel die Umsetzung der digitalen Barrierefreiheit in Desktopanwendungen von Webseiten bzw. webbasierte Anwendungen abweicht.

Für diesen Zweck werden die wesentlichen Eigenschaften der Web- und Desktopanwendungen betrachtet und es wird bewertet, ob die Eigenschaften einen Einfluss auf die Umsetzung der digitalen Barrierefreiheit in den Desktopanwendungen hat.

In der \cref{tab:Webanwendungen vs Desktopanwendung}\footnote{Eigene Darstellung} sind die Unterschiede wichtiger Eigenschaften\footnote{Cyber-Solutions \cite{CyberSolutions}} zwischen Web- und Desktopanwendungen betrachtet und es wird bewertet, ob die Eigenschaften die Umsetzung der digitalen Barrierefreiheit in den Desktopanwendungen verhindern können.

\begin{table}[H]
	\caption{Wesentliche Unterschiede zwischen  Web- und Desktopanwendungen}
	\label{tab:Webanwendungen vs Desktopanwendung}
	\centering
	\includegraphics[width=6in]{Webanwendungen vs Desktopanwendung.pdf}
\end{table}

Anhand der Auswertung der Eigenschaften in \cref{tab:Webanwendungen vs Desktopanwendung} werden die Normen der \ac{WCAG} 2.0 bzw. der \ac{WCAG} 2.1 genauso für Desktopanwendungen behandelt, wie für webbasierte Anwendungen. Es werden keine bestimmte Ausnahmen für die Desktopanwendungen gemacht, es sei denn, die Umsetzung einer bestimmten Richtlinie bzw. eines bestimmten Erfolgskriteriums ist ausschließlich für Webseiten gedacht und in den Desktopanwendungen nicht möglich ist.

\subsubsection{Umsetzbare Kriterien in den aktuellen Desktopanwendungen}
\label{subsec: Umsetzbare Kriterien}

Es wird nun die Umsetzbarkeit aller Richtlinien der \ac{WCAG} 2.0 bzw. der \ac{WCAG} 2.1 in den Desktopanwendung untersucht. Demzufolge werden Richtlinien oder nur bestimmte Erfolgskriterien ausgeschlossen. Eine Richtlinie gilt erst als umsetzbar in den Desktopanwendung, wenn alle Erfolgskriterien der Konformitätsstufe A erfüllt werden können. Außerdem wird  jede Entscheidung, ob eine Richtlinie bzw. ein Erfolgskriterium erfüllt werden kann, begründet. In \cref{subsec: Soll-Zustand der Desktopanwendungen} werden dementsprechend die in den Desktopanwendung umsetzbare Richtlinien präsentiert und es werden Techniken zur Beseitigung der digitalen Barrieren bezüglich der Erfolgskriterien vorgeschlagen, um die betroffene Richtlinien erfüllen zu können.

In der folgenden Tabelle werden alle Richtlinien und jedes ihrer Erfolgskriterien mit der entsprechenden Konformitätsstufe präsentiert und auf die Umsetzbarkeit in den Desktopanwendungen geprüft:
\\
\unsure{Falls doch die AKG-Software statt dem allgemeinen Fall betrachtet wird, dann wird die Einleitung hier spezifiziert}

\addcontentsline{lot}{table}{2 \hspace{1.5em} Bewertung der Richtlinien}
\includepdf[pages=-, scale=0.9, pagecommand={}]{Bewertung der Richtlinien}

\subsubsection{Barrierefreiheit im Software-Entwicklungsprozess}
Beachten die Entwickler die Umsetzung der digitalen Barrierefreiheit von Anfang an nicht, da der Kunde sie im Lastenheft gar nicht verlangt, dann wird eine Umarbeitung der Software sehr aufwendig sein, falls der Kunde später sie doch wünscht. Außerdem ist der Aufwand für das Entwickeln barrierefreier Software von Anfang an geringer als viele Entwickler glauben, da alle  derzeitigen modernen Betriebssysteme globale Funktionen zur Barrierefreiheit anbieten, die der Softwareentwickler dementsprechend nutzen kann wie Beispielsweise die Bildschirmtastatur, die kontrastreiche Darstellung, die Bildschirmlupe und das Vorlesen von Elementen usw.\footnote{Standards für barrierefreie Software \cite{DEVINSIDER}}


\subsection{Soll-Zustand der Desktopanwendungen}
\label{subsec: Soll-Zustand der Desktopanwendungen}
