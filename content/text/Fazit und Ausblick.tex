\section{Fazit \& Ausblick}

\subsection{Zusammenfassung der Arbeit}
Das Thema "`Digitale Barrierefreiheit"' ist seit den 1970er-Jahren eingefordert,\footnote{Accessibility über Desktopanwendungen hinaus–Barrierefreiheit \cite{buhler2017accessibility}} jedoch bekam das Thema seine rechtliche Dimension erst später durch die Normen der \ac{WCAG} sowie der \ac{BITV}. Die digitale Barrierefreiheit umfasst alle digitalen Inhalte, diese können Webseiten, PDFs, Desktopsoftware, Online-Redaktionen, Webkonferenzen, Online-Videos und Mobile-Anwendungen sein. Allerdings ist die Konzentration auf die Webseiten und webbasierte Anwendungen, die meistens digitale Angebote anbieten, stärker. Aus diesem Grund wurde in \cref{subsec: Kriterienvergleich der Barrierefreiheit zwischen Web- und Desktopanwendungen} ein Vergleich zwischen den wesentlichen Eigenschaften der Web- und Desktopanwendungen gezogen und mit dem Ergebnis dieses Vergleiches konnten die Richtlinien der \ac{WCAG} 2.0 bzw. der \ac{WCAG} 2.1 für die Umsetzung in den Desktopanwendungen bewertet werden. Der Nachteil der fehlenden digitalen Barrierefreiheit in jeder Software ist groß, da im schlimmsten Fall nicht nur der Nutzer der Software benachteiligt wird, sondern es werden ganze Zielgruppen ausgeschlossen, was keinem Unternehmen dient. Auf der anderen Seite sind die Vorteile einer barrierefreien Software nicht zu unterschätzen, denn eine barrierefreie Software erreicht mehr Menschen und stellt dem Nutzer keine Hindernisse in den Weg.

\subsection{Erfolgseinschätzung und Ausblick}
Das Ziel dieser Arbeit wurde erreicht, indem die erarbeiteten Kriterien der Standard-Richtlinien der \ac{WCAG} 2.0 bzw. der \ac{WCAG} 2.1 auf die Umsetzung der digitalen Barrierefreiheit in den Desktopanwendungen hin untersucht wurden und damit der Katalog der umsetzbaren Kriterien in \cref{subsec: Soll-Zustand der Desktopanwendungen} erstellt wurde. Für die umsetzbaren Kriterien wurden Maßnahmen in \cref{subsec: Soll-Zustand der Desktopanwendungen} für die Beseitigung der digitalen Barrieren in der aktuellen Software des Unternehmens vorgeschlagen. Diese kommen dementsprechend zum Einsatz in der aktuellen Software und für das Entwickeln neuer Software. Es bleibt jedoch offen, ob die Kriterien, die ausgeschlossen wurden, in der fernen Zukunft zur Anwendungen kommen. Zu einem späteren Zeitpunkt könnten Tools eingesetzt werden, um das Untersuchen der Software auf die Umsetzung der relevanten Kriterien hin zu erleichtern und um sicherzustellen, dass es bei der Umsetzung der Kriterien an nichts fehlt.

% Muss am Ende des Reintexts sein, um die Seitenanzahl davon in Hinweise zum  Umfang der Arbeit zu schreiben
\label{seitenreinschrifft}
