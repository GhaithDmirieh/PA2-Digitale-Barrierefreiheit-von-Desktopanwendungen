\section{Grundlagen}

\subsection{AGK Software Consulting GmbH}
Das Unternehmen AKG ist ein Software-Entwicklungs-Unternehmen. Im Unternehmen wird Software für die Straßenplanung hergestellt. Die hergestellte Software umfasst ein breites Spektrum, von der Straßenplanung bis zur Kostenberechnung. Die Software besteht aus vielen Modulen, die jeweils einen Fachbereich behandeln und basiert auf
einer CAD-Plattform. Als Basis kommen drei unterschiedliche Plattformen zum Einsatz. Bei AKG wird die Software kontinuierlich weiterentwickelt und in halbjährlichen
Software-Versionen für die Kunden veröffentlicht. In den Software-Versionen werden jeweils wichtige Korrekturen für Software-Fehler und neue Funktionalitäten eingeführt. Die Entwicklung der Software wird in den vier Firmensitzen von AKG durchgeführt. Neben dem Hauptfirmensitz in Heitersheim gibt es weitere Firmensitze in Berlin, Wien und Köln. Für den Software-Entwicklungsprozess sind mehrere Abteilungen notwendig, die es in jedem Firmensitz gibt. Dazu gehören die Entwicklungsabteilung, die Qualitätssicherungsabteilung und die Support-Abteilung. Zur Verwaltung der Software-Entwicklung wird ein \ac{TFS} standortübergreifend verwendet. Im \ac{TFS} werden die Software Versionen, der Quellcode, die Aufgaben und weitere Aspekte der Software-Entwicklung verwaltet. Zur Kommunikation verwenden die Mitarbeiter standortübergreifend die Anwendung „Microsoft Teams“ (MS Teams). In den neusten Software-Versionen sind trotz aller Maßnahmen immer einige Software-Fehler vorhanden. Diese Software-Fehler können von den Kunden bemerkt und an die Mitarbeiter von AKG gemeldet werden. Die gemeldeten Software-Fehler werden anschließend von den Mitarbeitern von AKG behoben und in der nächsten Software-Version für die Kunden veröffentlicht. Das Erfassen der Fehler läuft über die Support-Abteilung bei AKG ab. Die Kunden können über ein Ticketsystem Software-Fehler in Form von Tickets an die Support-Abteilung melden. Die Mitarbeiter der Support-Abteilung kümmern sich um die weitere Bearbeitung der Tickets. Entweder wird das Ticket durch die Mitarbeiter der Support-Abteilung gelöst oder an die Entwicklungsabteilung weitergeleitet, damit das Ticket in der Software behoben werden kann.

\subsubsection{Abteilungsvorstellung}
In der Entwicklungsabteilung beschäftigen sich die Mitarbeiter damit, die Software-Fehler zu korrigieren und neue Funktionalitäten umzusetzen. Jeder Mitarbeiter ist dabei für bestimmte Module in der Software zuständig. Pro Modul können mehrere Mitarbeiter und jeder Mitarbeiter kann für mehrere Module zuständig sein. Die Mitarbeiter der Qualitätssicherungsabteilung sind dafür zuständig, die umgesetzten Änderungen der Entwicklungsabteilung in der Software zu überprüfen. Jede Änderung wird durch einen Mitarbeiter aus der Qualitätssicherungsabteilung geprüft, bevor die Änderung an die Kunden veröffentlicht werden kann. Bei fehlerhafter Umsetzung einer Änderung werden die Entwickler zur Korrektur der Änderung aufgefordert.

\subsection{Definitionen}
\comingSoon{Jeder unverständlicher Begriff bekommt hier einen Unterabschnitt}

\subsubsection{Informations- und Kommunikationstechnik}
\ac{IKT}: Der Begriff entstand zur Beginn der 90er Jahre des vorherigen Jahrhunderts. Bereits in den 1990er war man der Auffassung, dass beide Bereiche zusammenwachsen würden. Die Informationstechnologie umfasste damals die Großrechner. Die Kommunikationstechnologie hingegen beschäftigte sich ausschließlich mit dem Fernsprechnetz. Seit etwa 2000 wird der Begriff häufig als Oberbegriff für alles verwendet, was mit Information und Kommunikation zu tun hat.

\subsubsection{Erfolgskriterium}
Ein Erfolgskriterium ist eine testbare Aussage, die entweder wahr oder falsch ist, wenn man sie auf konkrete Webinhalte anwendet. In dieser Arbeit sind die Erfolgskriterien zur Erfüllung der Barrierefreiheit der \ac{WCAG} gemeint.

\subsubsection{\ac{WAI}}
\comingSoon{Coming soon}

\subsubsection{\ac{W3C}}
\comingSoon{Coming soon}

\subsection{Grundlagen der verwendeten Methoden}
\comingSoon{Falls noch andere Methoden kommen, dann als Unterabschnitt ausführlich und abstrakt erklären}

\subsubsection{Nutzwertanalyse}

Ist ein qualitatives Verfahren der Entscheidungstheorie. Es geht um die Beurteilung der Effektivität durch Bewertung von Alternativen/Varianten über gewichtete Zieldimensionen und Nutzeneinstufung. Die jeweiligen Beiträge werden zur Erreichung des Ziels quantifiziert. Bei einer Nutzwertanalyse werden alle Kriterien, egal ob sie monetär bewertbar sind oder nicht, gleich behandelt, da sie in ein Punktsystem umgerechnet werden. Es besteht hohe Flexibilität als Vorteil.\footnote{Nutzwertanalyse, Dr. Kristina Birn \cite{Dr.KB-Projektmanagment}}

Alle Zielkriterien werden in einem kreativen Prozess bearbeitet und ausgewählt. Die Kriterien müssen der Anforderung gerecht werden, damit sichergestellt wird, dass sie auch zur Problemlösung beitragen. Jedes Kriterium muss dazu beitragen, dass alle Kriterien insgesamt das Gesamtproblem lösen. Zudem muss jedes Kriterium bewertbar sein, d.h. das sachliche oder fachliche Hintergrundwissen, um das Kriterium bewerten zu können, muss vorhanden sein. Darüber hinaus muss ein Kriterium eine Relevanz in Bezug auf die Bewertung der Alternativen aufweisen und es muss reproduzierbar sein, d.h. die Bewertung des Kriteriums sollte zu jeder Zeit gleich ausfallen. Schließlich müssen alle Kriterien gewichtet werden und die Summe aller Gewichtungen muss 100\% ergeben. Dafür werden die Wichtigkeiten der einzelnen Kriterien auf das gesamte Problem festgestellt. Das kann mittels eines Paar-Vergleichs erfolgen.\footnote{Nutzwertanalysen in Marketing und Vertrieb \cite{kuhnapfel2014nutzwertanalysen}}

In dieser Projektarbeit wird die Nutzwertanalyse eingesetzt, um zu entscheiden, welche Software-Anforderungen der digitalen Barrierefreiheit in den Desktopanwendungen der AKG Software Consulting GmbH umsetzbar sind. Jede Anforderung bekommt eine Bewertung in Bezug auf die Erfüllung des Kriteriums auf einer Skala von 0 bis 10. Diese Bewertung wird mit der Gewichtung multipliziert, um den Punktwert zu erhalten. Jede Anforderung, die einen bestimmten Punktwert erreicht, wird angenommen und umgesetzt. \comingSoon{Hier den genauen Punkt erwähnen, den eine Anforderung erreichen soll, um als umsetzbar zu zeichnen}

Eine exemplarische Nutzwertanalyse ist in \cref{table:Beispielhafte Nutzwertanalyse} dargestellt\footnote{Eigene Darstellung}.

\begin{table}[H]
\caption[Beispielhafte Nutzwertanalyse]{Beispielhafte Nutzwertanalyse}
\vspace{0,1cm}
\begin{tabular}{cc|c|c|c|c|c|c|}
	\cline{3-8}
	& & \multicolumn{2}{c|}{\textbf{Anforderung 1}} & \multicolumn{2}{c|}{\textbf{Anforderung 2}} & \multicolumn{2}{c|}{\textbf{Anforderung 3}} \\ \hline
	\multicolumn{1}{|l|}{\textbf{Kriterium}} & \textbf{Gewichtung} & \textbf{B} & \textbf{PW} & \textbf{B} & \textbf{PW} & \textbf{B} & \textbf{PW} \\ \hline
	\multicolumn{1}{|l|}{\textbf{Kriterium 1}} & 60\%  & 4 & 2,4 & 2 & 1,2 & 3  & 1,8 \\ \hline
	\multicolumn{1}{|l|}{\textbf{Kriterium 2}} & 30\%  & 7 & 2,1 & 9 & 2,8 & 5  & 1,5 \\ \hline
	\multicolumn{1}{|l|}{\textbf{Kriterium 3}} & 10\%  & 3 & 0,3 & 1 & 0,1 & 10 & 1   \\ \hline
	\multicolumn{1}{|c|}{\textbf{Summe}}       & 100\% &   & 4,8 &   & 4,1 &    & 4,3 \\ \cline{1-2} \cline{4-4} \cline{6-6} \cline{8-8} 
\end{tabular}
\label{table:Beispielhafte Nutzwertanalyse}
\end{table}

\subsection{Motivation}

Im Jahr 2019 lebten in Deutschland 7,9 Millionen schwerbehinderte Menschen. "´Körperliche Behinderungen hatten 58\% der schwerbehinderten Menschen: Bei 25\% waren die inneren Organe beziehungsweise Organsysteme betroffen. Bei 11\% waren Arme und/oder Beine in ihrer Funktion eingeschränkt, bei weiteren 10\% Wirbelsäule und Rumpf. In 4\% der Fälle lag Blindheit beziehungsweise eine Sehbehinderung vor. Ebenfalls 4\% litten unter Schwerhörigkeit, Gleichgewichts- oder Sprachstörungen. Der Verlust einer oder beider Brüste war bei 2\% Grund für die Schwerbehinderung."' \footnote{Destatis. Statistisched Bundesamt \cite{DESTATIS}}

Diese Menschen können an der digitalen Welt nicht teilhaben, da sie auf digitale Barrierefreiheit angewiesen sind. Andere haben es schwer, denn die digitalen Barrieren nutzen keinem. Jeder wird sich über Software, mit der man problemlos klarkommen kann, freuen. Demzufolge ist der Abbau der digitalen Barrieren für alle gut. Kurzgefasst ist die Barrierefreiheit für Menschen mit Behinderungen ein Muss, sonst können sie die digitalen Software gar nicht nutzen. Ältere Menschen und Menschen, die beispielsweise nicht gut sehen, lesen oder sich konzentrieren können, profitieren von barrierefreien Software. Fachleute sprechen in diesem Zusammenhang von einer "´Usability"' im Deutschen "´Benutzerfreundlichkeit"' \footnote{Bayerische Staatsregierung \cite{BS}}
\\
\unsure{Erweitern}