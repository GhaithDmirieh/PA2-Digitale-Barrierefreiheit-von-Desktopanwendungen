\section{Grundlagen}
\label{sec:Grundlagen}

\subsection{AKG Software Consulting GmbH}
Das Unternehmen AKG ist ein Software-Entwicklungs-Unternehmen. Im Unternehmen wird Software für die Straßenplanung hergestellt. Die hergestellte Software umfasst ein breites Spektrum, von der Straßenplanung bis zur Kostenberechnung. Die Software besteht aus vielen Modulen, die jeweils einen Fachbereich behandeln und basiert auf
einer CAD-Plattform. Als Basis kommen drei unterschiedliche Plattformen zum Einsatz. Bei AKG wird die Software kontinuierlich weiterentwickelt und in halbjährlichen
Software-Versionen für die Kunden veröffentlicht. In den Software-Versionen werden jeweils wichtige Korrekturen für Software-Fehler und neue Funktionalitäten eingeführt. Die Entwicklung der Software wird in den sieben Firmensitzen von AKG durchgeführt. Neben dem Hauptfirmensitz in Heitersheim gibt es weitere Firmensitze in Berlin, Wien, Köln, Hamburg, Halle und Landquart. Für den Software-Entwicklungsprozess sind mehrere Abteilungen notwendig, die es in jedem Firmensitz gibt. Dazu gehören die Entwicklungsabteilung, die Qualitätssicherungsabteilung und die Support-Abteilung. Zur Verwaltung der Software-Entwicklung wird ein \ac{TFS} standortübergreifend verwendet. Im \ac{TFS} werden die Software-Versionen, der Quellcode, die Aufgaben und weitere Aspekte der Software-Entwicklung verwaltet. Zur Kommunikation verwenden die Mitarbeiter standortübergreifend die Anwendung „Microsoft Teams“ (MS Teams). In den neuesten Software-Versionen sind trotz aller Maßnahmen immer einige Software-Fehler vorhanden. Diese Software-Fehler können von den Kunden bemerkt und an die Mitarbeiter von AKG gemeldet werden. Die gemeldeten Software-Fehler werden anschließend von den Mitarbeitern von AKG behoben und in der nächsten Software-Version für die Kunden veröffentlicht. Das Erfassen der Fehler läuft über die Support-Abteilung bei AKG ab. Die Kunden können über ein Ticketsystem Software-Fehler in Form von Tickets an die Support-Abteilung melden. Die Mitarbeiter der Support-Abteilung kümmern sich um die weitere Bearbeitung der Tickets. Entweder wird das Ticket durch die Mitarbeiter der Support-Abteilung gelöst oder an die Entwicklungsabteilung weitergeleitet, damit das Ticket in der Software behoben werden kann.

\subsection{Abteilungsvorstellung}
In der Entwicklungsabteilung beschäftigen sich die Mitarbeiter damit, die Software-Fehler zu korrigieren und neue Funktionalitäten umzusetzen. Jeder Mitarbeiter ist dabei für bestimmte Module in der Software zuständig. Pro Modul können mehrere Mitarbeiter und jeder Mitarbeiter kann für mehrere Module zuständig sein. Die Mitarbeiter der Qualitätssicherungsabteilung sind dafür zuständig, die umgesetzten Änderungen der Entwicklungsabteilung in der Software zu überprüfen. Jede Änderung wird durch einen Mitarbeiter aus der Qualitätssicherungsabteilung geprüft, bevor die Änderung an die Kunden veröffentlicht werden kann. Bei fehlerhafter Umsetzung einer Änderung werden die Entwickler zur Korrektur der Änderung aufgefordert. Außerdem gibt es die Style-Guide-Gruppe, die aus einigen Personen aus verschiedenen Abteilungen besteht. Die Style-Guide-Gruppe beschäftigt sich mit Regelungen zur visuellen Gestaltung von Dialogen der Desktopanwendungen. In diesem Zusammenhang ist die Dokumentationsabteilung auch relevant, die sich mit Hilfe-Texten, Übersetzungen und \ac{GUI}-Design beschäftigt.

\subsection{Definitionen}
\label{subsec: Definitionen}

\subsubsection{CAPTCHA}
\begin{figure}[h]
	\begin{minipage}{0.5\textwidth}
		Steht für "`Completely Automated Public Turing test to tell Computers and Humans Apart"' im Deutschen ein vollständig automatisierter öffentlicher
		Turing-Test, um einen Menschen vom Computer zu unterscheiden.\footnote{Captcha security: A case study \cite{yan2009captcha}}
		% \caption{Der Text}
		% \label{Text}
	\end{minipage}
	\hfill
	\begin{minipage}{0.4\textwidth}
		% \textwidth bezieht sich nun auf die Minipage
		\includegraphics[width=\textwidth]{CAPTCHA.png}
		\caption[Beispiel-CAPTCHA]{Beispiel-CAPTCHA. \\Quelle: \cite{yan2009captcha}}
		%\label{Bild} 
	\end{minipage}
	% \caption{noch eine Caption für die Figur}
\end{figure}

\subsubsection{Erfolgskriterium}
Ein Erfolgskriterium ist eine testbare Aussage, die entweder wahr oder falsch ist, wenn man sie auf konkrete Webinhalte anwendet. In dieser
Arbeit sind die Erfolgskriterien zur Erfüllung der Barrierefreiheit der \ac{WCAG} gemeint.\footnote{Web Content Accessibility Guidelines 2.0 \cite{WCAG2.0}}

\subsubsection{\ac{WAI}}
Die \ac{WAI} ist eine Initiative der \ac{W3C}. Die \ac{WAI} entwickelt Standards und Hilfsmittel, die beim Verstehen und bei der Umsetzung 
der Barrierefreiheit helfen.\footnote{The World Wide Web Consortium (W3C) \cite{w3c}}

\subsubsection{\ac{W3C}}
Das \ac{W3C} ist eine internationale Community, die die Standards entwickelt, um das langfristige Wachstum des Web 
sicherzustellen.\footnote{The World Wide Web Consortium (W3C) \cite{w3c}}

\subsubsection{Audiodeskription}
Audiodeskription ist eine zur Tonspur zusätzlich hinzugefügte Schilderung, um visuelle Details zu beschreiben, die allein durch die Haupt-Tonspur 
nicht verständlich sind.\footnote{Web Content Accessibility Guidelines 2.0 \cite{WCAG2.0}}

\subsubsection{Erweiterte Audiodeskription}
Die erweiterte Audiodeskription wird zu audiovisuellen Inhalten hinzugefügt, indem das Video angehalten wird. Diese Zeit wird verwendet, 
um eine zusätzliche Audiodeskription hinzuzufügen. Eine erweiterte Audiodeskription wird nur dann benutzt, wenn der Sinn des Videos ohne diese verloren geht und wenn 
die Pausen zwischen den Schilderungen zu kurz sind, so dass die normalen Audiodeskriptionen nicht den Sinn des Inhaltes vermitteln 
können.\footnote{Web Content Accessibility Guidelines 2.0 \cite{WCAG2.0}}

\subsubsection{Synchronisierte Medien}
"`Audio oder Video, synchronisiert mit einem anderen Format für die Präsentation von Informationen und/oder mit zeitbasierten interaktiven 
Komponenten, es sei denn, das Medium ist eine Medienalternative für Text, die als solche deutlich gekennzeichnet 
ist."'\footnote{Web Content Accessibility Guidelines 2.0 \cite{WCAG2.0}}

\subsubsection{Durch Software bestimmbar}
Die vom Autor gelieferten Daten sollten so bereitgestellt werden, dass Benutzeragenten und assistierende Techniken diese Daten entnehmen können und in verschiedenen
Formen präsentieren können.\footnote{Web Content Accessibility Guidelines 2.0 \cite{WCAG2.0}}

\subsection{Motivation}
\label{subsec: Motivation}

Im Jahr 2019 lebten in Deutschland 7,9 Millionen schwerbehinderte Menschen. "`Körperliche Behinderungen hatten 58\% der schwerbehinderten Menschen: Bei 25\% waren die inneren Organe beziehungsweise Organsysteme betroffen. Bei 11\% waren Arme und/oder Beine in ihrer Funktion eingeschränkt, bei weiteren 10\% Wirbelsäule und Rumpf. In 4\% der Fälle lag Blindheit beziehungsweise eine Sehbehinderung vor. Ebenfalls 4\% litten unter Schwerhörigkeit, Gleichgewichts- oder Sprachstörungen. Der Verlust einer oder beider Brüste war bei 2\% Grund für die Schwerbehinderung."' \footnote{Destatis. Statistisched Bundesamt \cite{DESTATIS}}

Diese Menschen können an der digitalen Welt nicht teilhaben, da sie auf digitale Barrierefreiheit angewiesen sind. Andere haben es schwer, denn die digitalen Barrieren nutzen keinem. Jeder wird sich über Software, mit der man problemlos klarkommen kann, freuen. Demzufolge ist der Abbau der digitalen Barrieren für alle gut. Kurzgefasst ist die Barrierefreiheit für Menschen mit Behinderungen ein Muss, sonst können sie die digitale Software gar nicht nutzen. Ältere Menschen und Menschen, die beispielsweise nicht gut sehen, lesen oder sich konzentrieren können, profitieren von barrierefreier Software. Fachleute sprechen in diesem Zusammenhang von einer "`Usability"', im Deutschen "`Benutzerfreundlichkeit"' \footnote{Bayerische Staatsregierung \cite{BS}}

Nun betrachten wir die Tatsache aus politischer Sicht: Wenn immer mehr Informationen und Dienste über die neuen Medien angeboten werden, dann besteht die Gefahr, dass viele digitale Dienste nicht für alle Bevölkerungsgruppen gleichermaßen zugänglich und nutzbar sind.\footnote{Das eInclusion EU Projekt--Politik zur digitalen Integration und Barrierefreiheit in Europa \cite{redingeinclusion}}