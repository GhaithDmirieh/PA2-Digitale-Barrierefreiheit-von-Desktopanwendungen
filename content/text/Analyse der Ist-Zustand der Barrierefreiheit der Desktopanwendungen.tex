\section{Analyse der Ist-Zustand der Barrierefreiheit der Desktopanwendungen}
\unsure{Interne Informationen über die AKG-Software}

\subsection{Barrierefreiheit in der Software-Entwicklung}
Blinde Menschen können keine Computermaus bedienen, d.h. dass die Software komplett per Tastatur bedienbar sein soll. Ein weiterer Grund, wieso Softwareunternehmen sich ebenfalls um digitale Barrierefreiheit kümmern sollten. Es ist eine harte Anforderung an Softwareentwickler, da bezüglich des erwähnten Beispiels immer üblich war, die Software per Maus zu bedienen. Davon profitieren aber in dem Fall nicht nur Menschen mit Behinderungen. Funktioniert die Maus nicht mehr, so kann die Arbeit mit der Tastatur weitergehen. Programme, die eine Vorlesesoftware anbieten, können nur dann die graphische Oberfläche vorlesen, wenn sie mit Texten beschrieben wurde. Sogar Bilder sind davon betroffen. Ist die Oberfläche nicht gut beschriftet, so kann ein blinder Mensch oder auch Menschen mit Sehschwierigkeiten nichts erkennen. Für Menschen mit Seheinschränkungen ist beispielsweise ein Textcursor, der lediglich als senkrechter Strich im Eingabefeld dargestellt ist, nur schwer zu erkennen. Tastenkombinationen sind auch Barrieren, die für Menschen, die behinderungsbedingt nur eine Hand für das Bedienen der Software einsetzen können, die man durch die Individualisierbarkeit dieser Tastenkombinationen vermeiden kann. Individualisierbarkeit ist auch ein Begriff für Menschen mit Farbfehlsichtigkeit. Softwareunternehmen sollten die Möglichkeit anbieten, die Farben innerhalb der Software zu ändern. Da nicht jedes Unternehmen ihre Software auf digitale Barrierefreiheit testet, haben Menschen mit Behinderungen sehr geringe Auswahlmöglichkeiten.\footnote{PC Welt von IDG \cite{PcWelt}}

\subsubsection{Barrierefreiheit und Behindertengerecht}
\label{subsec:Barrierefreiheit und Behindertengerecht}

\comingSoon{Coming soon}

\subsubsection{Konformitätsstufen anhand der \ac{WCAG}}
\label{subsubsec: Konformitätsstufen}
Nach den Richtlinien der \ac{WCAG} für Barrierefreiheit unterscheidet sich die Umsetzung der Anforderungen an barrierefreie Software in verschiedene Niveaus, die in drei Konformitätsstufen abgestuft sind. Diese sind:

\begin{description}
	\item [Konformitätsstufen A:] Die minimale Konformitätsstufe. Es ist ein Muss, da Menschen sonst die Inhalte nicht wahrnehmen und bedienen können. Wenn
	Erfolgskriterien der Konformitätsstufe A verletzt werden, dann ist mindestens eine Nutzergruppe von der Nutzung ausgeschlossen.
	
	\item [Konformitätsstufen AA:] Sie ist in Europa und in Deutschland für öffentliche Stellen vorgegeben. Für eine Konformität auf dieser Stufe 
	müssen all Erfolgskriterien der Stufe \textbf{A} und der Stufe \textbf{AA} erfüllt werden. Sind die Erfolgskriterien dieser Stufe nicht umsetzbar, so
	müssen Alternativen dieser Erfolgskriterien
	zur Verfügung gestellt werden. Diese Stufe kann man als angestrebte Norm bezeichnet werden, da Menschen sonst die Inhalte nur schwer wahrnehmen und bedienen können.
	
	\item [Konformitätsstufen AAA:] Um auf diese Stufe zu kommen, müssen zuerst die ersten zwei Stufen \textbf{A} und \textbf{AA} erfüllt werden. "`Kein 
	Webauftritt wird jemals realistischerweise alle WCAG Erfolgskriterien auch der Konformitätsstufe AAA erfüllen. Nicht einmal die WAI empfiehlt, sich 
	dieses Ziel vorzunehmen. Es wird nicht empfohlen, Konformität auf Stufe AAA als allgemeine Richtlinie für komplette Websites zu fordern, da es bei manchen
	Inhalten nicht möglich ist, alle Erfolgskriterien der Stufe AAA zu erfüllen."'\footnote{Zweiter Blick \cite{ZweiterBlick}}
\end{description}

Zur besseren Vorstellung dieser Konformitätsstufen hier \cref{fig:Konformitätsstufen} \footnote{Hochschule für Technik und Wirtschaft Dresden \cite{HV}} beitragen.

\begin{figure}[H]
	\centering
	\includegraphics[width=1.0\textwidth]{Konformitätsstufen}
	\caption[Konformitätsstufen nach der \ac{BITV} 2.0]{Konformitätsstufen nach der \ac{BITV} 2.0}
	\label{fig:Konformitätsstufen}
\end{figure}

\vspace{2cm}

Es bestehen jedoch Ausnahmen für Technologien, die nicht barrierefrei sind und trotzdem werden unter der folgende Bedingungen eingesetzt dürfen:
\begin{itemize}
	\item Falls die Inhalte parallel in einer barrierefreien Version zur Verfügung gestellt werden können.
	\item Es dürfen Elemente eingesetzt werden, die nicht den Anforderungen an Barrierefreiheit entsprechen. Solange die Anforderungen der Barrierefreiheit erfüllt 
	sind, können andere Elemente die Wahrnehmung, Bedienbarkeit oder das Verständnis nicht stören.
\end{itemize}

\subsubsection{Normen der \ac{WCAG} 2.0}

\subsubsection{Normen der \ac{BITV} 2.0}


\subsubsection{Normen der \ac{WCAG} 2.1 als Erweiterung}
\label{subsec: Normen der WCAG 2.1}
