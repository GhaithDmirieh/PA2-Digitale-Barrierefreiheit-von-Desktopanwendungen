\addsec{Anhang}

\subsection*{Anlage 1}
\label{subsec: Anlage1}

\textbf{Allgemeine Grenzwerte zu Blitzen und roten Blitzen (general flash and red flash thresholds)}\footnote{Web Content Accessibility Guidelines 2.0 \cite{WCAG2.0}}

\begin{enumerate}
	\item Ein Blitz oder eine schnell wechselnde Sequenz von Bildern ist unterhalb des Grenzwertes (d.h. Inhalt besteht die Prüfung), wenn eines der Folgenden zutrifft:
	Es gibt nicht mehr als drei allgemeine Blitze und / oder nicht mehr als drei rote Blitze innerhalb von beliebigen Ein-Sekunden-Zeiträumen; oder

	\item Der zusammengenommene Bereich von gleichzeitig auftretenden Blitzen belegt nicht mehr als eine Summe von .006 Steradianten innerhalb jedes beliebigen 10 Grad 		visuellen Feldes auf dem Bildschirm (25\% jedes beliebigen 10-Grad visuellen Feldes auf dem Bildschirm) bei einer typischen Betrachtungsentfernung. Wobei:
	\begin{itemize}
		\item Ein allgemeiner Blitz als ein Paar von entgegengesetzten Änderungen in relativer Luminanz von 10\% oder mehr der maximalen relativen Luminanz definiert wird, 		wobei die relative Luminanz des dunkleren Bildes unter 0.80 liegt; und wo „ein Paar von entgegengesetzten Änderungen“ eine Zunahme gefolgt von einer Abnahme 				ist oder eine Abnahme gefolgt von einer Zunahme. Und
		\item Ein roter Blitz als jedes Paar von entgegengesetzten Übergängen, bei denen ein gesättigtes Rot beteiligt ist, definiert wird.
	\end{itemize}
\end{enumerate}

\textbf{Ausnahme:} Blitzen, das ein feines, ausgeglichenes Muster wie weißes Rauschen oder ein wechselndes Schachbrettmuster mit "`Quadraten"' ist, die auf einer Seite kleiner sind als 0.1 Grad (des visuellen Feldes bei typischem Betrachtungsabstand), verstößt nicht gegen die Grenzwerte.

\textbf{Anmerkung 1:} Die Benutzung eines 341 x 256 Pixel großen Rechtecks irgendwo in dem gezeigten Bildschirmbereich, wenn der Inhalt bei 1024 x 768 Pixeln betrachtet wird, gibt bei allgemeiner Software oder Webinhalten eine gute Einschätzung eines 10 Grad visuellen Feldes für Standard-Bildschirmgrößen und Betrachtungsentfernungen (z.B. 15-17 Zoll Bildschirm bei 56 - 66 cm). (Bildschirme mit höheren Auflösungen, auf denen das gleiche Rendering des Inhalts gezeigt wird, ergeben kleinere und sicherere Bilder, daher werden die geringeren Auflösungen benutzt, um die Grenzwerte zu definieren.)

\textbf{Anmerkung 2:} Ein Übergang ist der Wechsel in relativer Luminanz (oder relativer Luminanz/Farbe für rotes Blitzen) zwischen nebeneinander liegenden Spitzen und Senken in einem Graph von relativer Luminanzmessung (oder relativer Luminanz/Farbe bei rotem Blitzen) in Bezug auf die Zeit. Ein Blitz besteht aus zwei entgegengesetzten Änderungen.

\textbf{Anmerkung 3:} Die derzeitige Arbeitsdefinition in diesem Fachbereich für ein "`Paar von entgegengesetzten Übergängen, die ein gesättigtes Rot beinhalten"' lautet: Wenn für jeden oder beide Zustände, die in jedem Übergang involviert sind, R/(R+ G + B) >= 0.8 und der Wechsel des Wertes von (R-G-B)x320 > 20 (negative Werte von (R-G-B)x320 werden auf Null gesetzt) für beide Übergänge ist. R, G, B Werte reichen von 0-1 wie in der "`relativen Luminanz"'-Definition festgelegt. [HARDING-BINNIE]

\textbf{Anmerkung 4:} Es gibt Werkzeuge, welche die Analyse durch die Erfassung des Video-Bildschirms ausführen. Es ist allerdings kein Werkzeug nötig, um diese Bedingung zu evaluieren, wenn das Blitzen weniger oder gleich 3 Blitze pro Sekunde ist. Der Inhalt besteht automatisch die Prüfung (siehe \# 1 und \# 2 oben).

\appendix
