\addsec{Kurzfassung}

In dieser Projektarbeit wird die Thematik behandelt um die Untersuchung der digitalen Barrierefreiheit in Dialogen der Desktopanwendungen. Es wird geprüft, inwiefern die Umsetzbarkeit der Standards der digitalen Barrierefreiheit in Dialogen der Desktopanwendungen möglich ist. Für den Zweck werden Normen der \ac{WCAG} 2.0 bzw. der \ac{WCAG} 2.1 betrachtet. Die Richtlinien der erwähnten Normen werden anhand ihrer Erfolgskriterien ausgewertet.

Als Resultat dieser wissenschaftlichen Arbeit wird ein Katalog der in den Desktop Dialogen umsetzbaren Kriterien des Standards der digitalen Barrierefreiheit erstellt. Dieser wird dementsprechend nicht nur für den Abbau von digitalen Barrieren der aktuellen Software verwendet, sondern auch bei der Entwicklung neuer Software berücksichtigt.

Die AKG-Software besitzt als Hilfsmittel zu ihrer Software eine webbasierte Dokumentation, die jeden Dialog der Software beschreibt und alle Funktionalitäten dieses Dialogs erklärt. Nichtsdestotrotz wird die Software nicht als barrierefrei bezeichnet, falls die Dokumentation barrierefrei ist. Infolgedessen wird in dieser Arbeit nicht untersucht, ob die webbasierte Dokumentation barrierefrei ist.